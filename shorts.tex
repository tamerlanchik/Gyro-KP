\documentclass[main.tex]{subfiles}
\begin{document}
\section*{Используемые обозначения и сокращения}
\begin{table}[h]
    \begin{center}
        \textbf{Сокращения}
        \begin{tabular}{p{0.1\linewidth}p{0.9\linewidth}}
            ДД & динамический демпфер \\
            ПФ & передаточная функция \\
            ОДЗ & область допустимых значений (функции) \\
        \end{tabular}
        \textbf{Обозначения}
        \begin{tabular}{p{0.1\linewidth}p{0.9\linewidth}}
            \( x, y \) & оси внутренней и наружной рамок соответственно \\
            \( \alpha, \beta \) & углы поворота вокруг осей y и x \\
            \( \alpha_1 \) & угол поворота ДД  по оси y \\
            \( A \) & суммарный момент инерции системы относительно оси наружной рамки y \\
            \( A_\text{ДД} \) & момент инерции маховика динамического демпфера относительно оси y \\ 
            \( B \) & суммарный момент инерции системы относительно оси внутренней рамки x \\
            H & кинематический момент ротора гироскопа \\
            \( \mu_\alpha,\ \mu_\beta \) & коэффициенты вязкого трения в осях наружной и внутренней рамок \\
            \( K_{\text{ОС}} \) & коэффициент обратной связи \\
            \( M_\alpha,\ M_\beta \) & внешние моменты, действующие на систему по осям y и x \\
            \( \varphi(\dot{\alpha}),\ \eta \) & функция и коэффициент нелинейности сухого трения в оси наружной рамки \\
            \( C,\ \mu \) & коэффициенты упругой и диссипативной связей \\
            \( M_\text{ДД} \) & момент упруго-диссипативного взаимодействия кожуха курсового
            гироскопа с инерционной массой динамического демпфера \\
        \end{tabular}
    \end{center}    
\end{table}
\end{document}